\documentclass[brazil,english]{brthesis}



% ---
% Informações de dados para CAPA e FOLHA DE ROSTO
% ---
\titulo{An example of a thesis written with \textsf{brthesis}}
\autor{Fábio Pinto Fortkamp}
\local{Florianópolis}
\data{December 2015}
\orientador{My Advisor}

\instituicao{%
  Universidade do Brasil
  \par
  Programa de Pós-Graduação em Engenharia Mecânica}
\tipotrabalho{Ph.D. Thesis}
% O preambulo deve conter o tipo do trabalho, o objetivo, 
% o nome da instituição e a área de concentração 
\preambulo{Ph.D. thesis to be submitted}
% ---


% ---
% Configurações de aparência do PDF final

% alterando o aspecto da cor azul
\definecolor{blue}{RGB}{41,5,195}

% informações do PDF
\makeatletter
\hypersetup{
     	%pagebackref=true,
		pdftitle={\@title}, 
		pdfauthor={\@author},
    	pdfsubject={\imprimirpreambulo},
	    pdfcreator={LaTeX with abnTeX2},
		pdfkeywords={abnt}{latex}{abntex}{abntex2}{trabalho acadêmico}, 
		colorlinks=true,       		% false: boxed links; true: colored links
    	linkcolor=black,          	% color of internal links
    	citecolor=black,        		% color of links to bibliography
    	filecolor=black,      		% color of file links
		urlcolor=black,
		bookmarksdepth=4
}
\makeatother
% --- 

% ---
% compila o indice
% ---
\makeindex
% ---

\graphicspath{{fig/}}

% ----
% Início do documento
% ----
\begin{document}

% Retira espaço extra obsoleto entre as frases.
\frenchspacing 

% ----------------------------------------------------------
% ELEMENTOS PRÉ-TEXTUAIS
% ----------------------------------------------------------
% \pretextual

% ---
% Capa
% ---
\imprimircapa
% ---

%% ---
%% Folha de rosto
%% (o * indica que haverá a ficha bibliográfica)
%% ---
%\imprimirfolhaderosto*
%% ---

% ---
% Inserir a ficha bibliografica
% ---

% Isto é um exemplo de Ficha Catalográfica, ou ``Dados internacionais de
% catalogação-na-publicação''. Você pode utilizar este modelo como referência. 
% Porém, provavelmente a biblioteca da sua universidade lhe fornecerá um PDF
% com a ficha catalográfica definitiva após a defesa do trabalho. Quando estiver
% com o documento, salve-o como PDF no diretório do seu projeto e substitua todo
% o conteúdo de implementação deste arquivo pelo comando abaixo:
%
% \begin{fichacatalografica}
%     \includepdf{fig_ficha_catalografica.pdf}
% \end{fichacatalografica}

% \include{Ficha_catalografica}
% ---

% ---
% Inserir folha de aprovação
% ---

% Isto é um exemplo de Folha de aprovação, elemento obrigatório da NBR
% 14724/2011 (seção 4.2.1.3). Você pode utilizar este modelo até a aprovação
% do trabalho. Após isso, substitua todo o conteúdo deste arquivo por uma
% imagem da página assinada pela banca com o comando abaixo:
%
% \includepdf{folhadeaprovacao_final.pdf}
%
%

%\include{Folha_de_aprovacao}
% ---

% ---
% Dedicatória
% ---
%\include{Dedicatoria}
% ---

% ---
% Agradecimentos
% ---
%\include{Agradecimentos}
% ---

% ---
% Epígrafe
% ---
%\include{Epigrafe}
% ---

% ---
% RESUMOS
% ---

% resumo em português
%\include{Resumo}

% resumo em inglês
%\include{Abstract}
% ---

% ---
% inserir lista de ilustrações
% ---
\pdfbookmark[0]{\listfigurename}{lof}
\listoffigures*
\cleardoublepage
% ---

% ---
% inserir lista de tabelas
% ---
\pdfbookmark[0]{\listtablename}{lot}
\listoftables*
\cleardoublepage
% ---


% ---
% inserir lista de símbolos
\renewcommand{\nomname}{\listadesimbolosname}
\pdfbookmark[0]{\nomname}{las}
\printnomenclature
%\printglossary
\cleardoublepage

% ---
% inserir o sumario
% ---
\pdfbookmark[0]{\contentsname}{toc}
\tableofcontents*
\cleardoublepage
% ---

% ----------------------------------------------------------
% ELEMENTOS TEXTUAIS
% ----------------------------------------------------------
\textual

\chapter{A chapter title}
\label{cha:chapter-title}

Let's beginning with an equation:

\begin{equation}
  \label{eq:1}
  c^2 = a^2 + b^2
\end{equation}

The reference can be produced by \textsl{autoref}: \autoref{eq:1}.

\section{A section title}
\label{sec:section-title}

In english, numbers should be printed with a dot as a decimal separator, like in $\num{3.14}$ or $\SI{1.3}{\pascal}$

\subsection{A subsection}
\label{sec:subsection}

See for example \autoref{fig:empty}.

\begin{figure}[!ht]
  \centering
  
  \caption{An empty figure}
  \label{fig:empty}
\end{figure}

% ----------------------------------------------------------
% ELEMENTOS PÓS-TEXTUAIS
% ----------------------------------------------------------
\postextual

% ----------------------------------------------------------
% Referências bibliográficas
% ----------------------------------------------------------
%\bibliography{refs/Thermo-Foam-Ref}



\end{document}


%%% Local Variables:
%%% mode: latex
%%% TeX-master: t
%%% End:
